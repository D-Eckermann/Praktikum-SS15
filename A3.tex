%% LyX 2.1.2 created this file.  For more info, see http://www.lyx.org/.
%% Do not edit unless you really know what you are doing.
\documentclass[twoside,ngerman]{scrartcl}
\usepackage{mathpazo}
\usepackage[T1]{fontenc}
\usepackage[latin9]{inputenc}
\usepackage[a4paper]{geometry}
\geometry{verbose,tmargin=2cm,bmargin=25mm,lmargin=20mm,rmargin=10mm}
\usepackage{fancyhdr}
\pagestyle{fancy}
\usepackage{babel}
\usepackage{amsmath}
\usepackage{amssymb}
\usepackage[unicode=true,pdfusetitle,
 bookmarks=true,bookmarksnumbered=true,bookmarksopen=false,
 breaklinks=false,pdfborder={0 0 1},backref=false,colorlinks=false]
 {hyperref}
\usepackage{breakurl}

\makeatletter
%%%%%%%%%%%%%%%%%%%%%%%%%%%%%% Textclass specific LaTeX commands.
\numberwithin{equation}{section}

%%%%%%%%%%%%%%%%%%%%%%%%%%%%%% User specified LaTeX commands.
\usepackage{pgfplots}
\pgfplotsset{width=7cm}

\makeatother

\begin{document}

\title{Versuchsprotokoll}


\subtitle{Versuch A3:\\
Absorption von $\gamma-$und $\beta-$Strahlung}


\date{\{Datum\}}


\author{Gruppe 6MO:\\
Frederik Edens\\
Dennis Eckermann}

\maketitle
\vfill{}


\tableofcontents{}

\vfill{}


\newpage{}


\section{Einleitung}


\subsection{Arten von Strahlung}

In diesem Versuch werden $\beta-$und $\gamma-$Zerf�lle gemessen.

Diese sind zwei von drei Arten den radioaktiven Zerfalls, die Dritte
Art ist die $\alpha-$Strahlung. Bei der $\alpha-$Strahlung handelt
es sich um Heliumatomkerne, bestehend aus zwei Protonen und zwei Neutronen,
die aus dem radioaktiven Atom hinausgeschleudert werden, dabei nimmt
die Masse des Atoms um vier atomare Masseneinheiten ab (4 atomare
Masseneinheiten=4u$\thickapprox6,64\cdot10^{-27}kg$) und die Ordnungszahl
wird um zwei verringert, dadurch entsteht ein neues Element. Die $\alpha$-
Strahlung besitzt diskrete Energiebetr�ge.

Da die $\alpha-$Strahlung nicht in diesem Versuch vorkommt, wird
nicht weiter darauf eingegangen.

Bei der $\beta-$Strahlung gibt es zwei m�gliche Zerf�lle die stattfinden
k�nnen. Zum einen der $\beta^{-}$-Zerfall bei dem ein Neutron im
Atomkern zu einem Proton sowie Elektron und einem Neutrino zerf�llt,
dabei verbleibt das Proton im Atomkern, das Elektron ist das Teilchen,
welches als $\beta$-Strahlung gemessen werden kann, das Neutrino
ist nur sehr schwer nachweisbar. Neben dem $\beta^{-}$-Zerfall gibt
es noch den $\beta^{+}$-Zerfall bei dem ein Proton in ein Neutron,
ein Positron und ein Neutrino zerf�llt, das Positron kann als $\beta$-Strahlung
gemessen werden, auch hier ist das Neutrino nur sehr schwer nachweisbar.

Im Gegensatz zur $\alpha$- und $\gamma$- Strahlung besitzt die $\beta$-
Strahlung keine diskreten sondern kontinuierliche Energiebetr�ge bis
hin zu einer gewissen Maximalenergie, dieses Ph�nomen liegt an dem
Neutrino, welches die fehlende Energie aufnimmt.

Die $\gamma$- Strahlung besteht aus hochenergetischen Photonen, den
sogenannten $\gamma$- Quanten, diese entstehen aus angeregten Atomen
dadurch, dass die Protonen und Neutronen im Atomkern ihren Energiezustand
�ndern und die Energiedifferenz wird als $\gamma$- Quant ausgesendet,
diese haben wie die $\alpha$- Strahlung diskrete Energiebetr�ge.


\subsection{Absorption von Strahlung}


\subsubsection{Absorption von $\gamma$- Strahlung}

Bei der Absorption von $\gamma$- Strahlung gibt es drei Absorptionsmechanismen,
dem Photoeffekt bei welchem das $\gamma$-Quant von einem Atom absorbiert
wird und dieses emittiert ein Elektron auf einer der inneren Schalen
der Elektronenh�lle. Der zweite Absorptionsmechanismus ist der Comptoneffekt,
bei dem eine inelastische Streuung des $\gamma$- Quants stattfindet,
in welchem es die Frequenz und die Richtung �ndert. Der dritte Absorptionsmechanismus
ist die Paarbildung, dort entstehen aus einem $\gamma$- Quant hoher
Energie ( E>2$m_{e}c^{2}\thickapprox1MeV$) ein Elektron und ein Positron.

Der Absorptionskoeffizient $\mu$setzt sich aus allen drei Absorptionsmechanismen
zusammen
\begin{equation}
\mu=\mu_{Photo}+\mu_{Compton}+\mu_{Paar}
\end{equation}


diese Gr��e ist ebenfalls vom Absorbermaterial abh�ngig.

Treffen nun N $\gamma$- Quanten auf einen Absorber der Dicke \emph{dx
}so wird folgende Relation erhalten
\begin{equation}
dN=-\mu Ndx
\end{equation}


gelegentlich wird auch die Massenbelegung $\rho$eingef�hrt, welches
die Dichte des Absorbermaterials darstellt. Dadurch wird $\mu$zu
$\mu_{m}=\frac{\mu}{\rho}$, also 
\begin{equation}
dN=-\mu_{m}N\rho dx
\end{equation}


durch Integration ergibt sich eine exponentielle Abnahme der Teilchenzahl
mit der Absorberdicke.

\begin{equation}
N(x)=N_{0}\cdot exp(-\mu)=N_{0}\cdot exp(-\mu_{m}\rho x)
\end{equation}



\subsubsection{Absorption von $\beta-$Strahlung}

Die exponentielle Abnahme gilt nicht f�r $\beta-$Strahlung, der Grund
hierf�r ist, dass die $\beta-$Absorption nicht durch einen einzigen
Prozess erfolgt. Die Elektronen verlieren ihre Energie durch mehrere
inelastische St��e mit den Atomen des Absorbermaterials, auch ist
es m�glich, dass sie durch eine Richtungs�nderung aus dem Strahl ausscheiden.
Somit wird f�r $\beta-$Strahlung die ,,mittlere Reichweite`` $R_{m}$
definiert, dass ist die Absoberdicke bei der noch 50\% der Anfangsstrahlung
vorhanden ist, daneben wird die ,,praktische Reichweite`` definiert,
diese ist als Schnittpunkt des extrapolierten linearen Abfalls der
Reichweitenverteilung mit der Achse \emph{N(x)=0 }definiert.


\subsection{Nachweis der Strahlung}

Zum Nachweis der Strahlung wird ein Geiger-M�ller-Z�hlrohr verwendet.
Im luftdichten Z�hlrohr befindet sich Argon unter einem Druck von
ungef�hr 100mbar und Alkoholdampf, welches einen Partialdruck von
etwa 10mbar hat. Es wird praktisch jedes $\alpha$- und $\beta$-Teilchen
vom Z�hlrohr erfasst, die $\gamma$- Strahlen werden mit einer kleineren
Menge nachgewiesen.

Im wesentlichen wird der Nachweismechanismus durch eine Ionisierung
eines Argon Atoms, da sich im Inneren des Z�hlrohrs ein elektrisches
Feld befindet, wird das Ionenpaar, welches aus Elektron und dem Argon-Ion
besteht, jeweils zur Anode bzw. zur Kathode hin beschleunigt. 

Auf diesem Weg ionisieren das Elektron andere Atome und es kommt zu
einer Entladungslawine. Wenn das Argon-Ion auf die Z�hlrohrwand auftrifft,
werden Sekund�relektronen emittiert, diese tragen zur Erhaltung des
Entladungsprozesses bei. Die Alkoholmolek�le stoppen diesen Entladungsprozess.

Nach jedem Entladungssto� bleibt das Z�hlrohr f�r einige $10^{-4}s$
f�r neue Teilchen unempfindlich, diese Zeit wird ,,Totzeit`` genannt.
Erst nach diese Zeit kann ein neues Teilchen detektiert werden.


\subsection{Poisson-Verteilung}

Der Erfahrung nach ist die Rate mit der Radioaktive-Zerf�lle stattfinden
mit der Poisson-Verteilung beschreibbar. Die Poisson-Verteilung geht
aus der Binomial-Verteilung f�r eine gro�e Zahl von Objekten und einer
geringen Ereigniswahrscheinlichkeit hervor. Die allgemeine Gleichung
lautet 
\begin{equation}
\psi_{n}=\frac{\bar{k}^{k}\cdot e^{-\bar{k}}}{k!}\label{eq:Poisson-Verteilung allgemein}
\end{equation}


$\bar{k}$ entspricht dem Erwartungswert 
\begin{equation}
\bar{k}=np
\end{equation}


$n$ entspricht hier der Anzahl der Atome und $p$ entspricht der
Wahrscheinlichkeit f�r einen Zerfall.

Bei einer Poisson-Verteilung sind Erwartungswert und Varianz gleich.

\begin{equation}
\sigma^{2}=np=\bar{k}
\end{equation}


Um nun die Poisson-Verteilung auf den radioaktiven Zerfall anzuwenden
muss der Mittelwert f�r die gemessenen Zerf�lle $\bar{N}$ bestimmt
werden. Dazu werden n-mal die Zahl der zerfallenen Kerne $N_{i}$
jeweils in einem festen Zeitintervall $\triangle T$ gemessen. Dann
wird aus \ref{eq:Poisson-Verteilung allgemein} 
\begin{equation}
\psi(N)=\frac{\bar{N}^{N}\cdot e^{-\bar{N}}}{N!}
\end{equation}


darauf ergibt sich f�r die mittlere Streuung,
\[
\sigma=\sqrt{\bar{N}}
\]


dieses wird $\sqrt{N}-Gesetz$ genannt.
\end{document}
