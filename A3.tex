%% LyX 2.1.2 created this file.  For more info, see http://www.lyx.org/.
%% Do not edit unless you really know what you are doing.
\documentclass[twoside,ngerman]{scrartcl}
\usepackage{mathpazo}
\usepackage[T1]{fontenc}
\usepackage[latin9]{inputenc}
\usepackage[a4paper]{geometry}
\geometry{verbose,tmargin=2cm,bmargin=25mm,lmargin=20mm,rmargin=10mm}
\usepackage{fancyhdr}
\pagestyle{fancy}
\usepackage{babel}
\usepackage{amsmath}
\usepackage{amssymb}
\usepackage[unicode=true,pdfusetitle,
 bookmarks=true,bookmarksnumbered=true,bookmarksopen=false,
 breaklinks=false,pdfborder={0 0 1},backref=false,colorlinks=false]
 {hyperref}
\usepackage{breakurl}

\makeatletter
%%%%%%%%%%%%%%%%%%%%%%%%%%%%%% Textclass specific LaTeX commands.
\numberwithin{equation}{section}

%%%%%%%%%%%%%%%%%%%%%%%%%%%%%% User specified LaTeX commands.
\usepackage{pgfplots}
\pgfplotsset{width=7cm}

\makeatother

\begin{document}

\title{Versuchsprotokoll}


\subtitle{Versuch A3:\\
Absorption von $\gamma-$und $\beta-$Strahlung}


\date{\{Datum\}}


\author{Gruppe 6MO:\\
Frederik Edens\\
Dennis Eckermann}

\maketitle
\vfill{}


\tableofcontents{}

\vfill{}


\newpage{}


\section{Einleitung}


\subsection{Arten von Strahlung}

In diesem Versuch werden $\beta-$und $\gamma-$Zerf�lle gemessen.

Diese sind zwei von drei Arten den radioaktiven Zerfalls, die Dritte
Art ist die $\alpha-$Strahlung. Bei der $\alpha-$Strahlung handelt
es sich um Heliumatomkerne, bestehend aus zwei Protonen und zwei Neutronen,
die aus dem radioaktiven Atom hinausgeschleudert werden, dabei nimmt
die Masse des Atoms um vier atomare Masseneinheiten ab (4 atomare
Masseneinheiten=4u$\thickapprox6,64\cdot10^{-27}kg$) und die Ordnungszahl
wird um zwei verringert, dadurch entsteht ein neues Element. Die $\alpha$-
Strahlung besitzt diskrete Energiebetr�ge.

Da die $\alpha-$Strahlung nicht in diesem Versuch vorkommt, wird
nicht weiter darauf eingegangen.

Bei der $\beta-$Strahlung gibt es zwei m�gliche Zerf�lle die stattfinden
k�nnen. Zum einen der $\beta^{-}$-Zerfall bei dem ein Neutron im
Atomkern zu einem Proton sowie Elektron und einem Neutrino zerf�llt,
dabei verbleibt das Proton im Atomkern, das Elektron ist das Teilchen,
welches als $\beta$-Strahlung gemessen werden kann, das Neutrino
ist nur sehr schwer nachweisbar. Neben dem $\beta^{-}$-Zerfall gibt
es noch den $\beta^{+}$-Zerfall bei dem ein Proton in ein Neutron,
ein Positron und ein Neutino zerf�llt, das Positron kann als $\beta$-Strahlung
gemessen werden, auch hier ist das Neutrino nur sehr schwer nachweisbar.

Im Gegensatz zur $\alpha$- und $\gamma$- Strahlung besitzt die $\beta$-
Strahlung keine diskreten sondern kontinuierliche Energiebetr�ge bis
hin zu einer gewissen Maximalenergie, dieses Ph�nomen liegt an dem
Neutrino, welches die fehlende Energie aufnimmt.

Die $\gamma$- Strahlung besteht aus hochenergetischen Photonen, den
sogenannten $\gamma$- Quanten, diese entstehen aus angeregten Atomen
dadurch, dass die Protonen und Neutronen im Atomkern ihren Energiezustand
�ndern und die Energiedifferenz wird als $\gamma$- Quant ausgesendet,
diese haben wie die $\alpha$- Strahlung diskrete Energiebetr�ge.


\subsection{Absorption von Strahlung}


\end{document}
